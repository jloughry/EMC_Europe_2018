%
% This is a template for a new research paper. For information on this file
% please contact Joe Loughry at Tel. +1 720 277 7800 (time zone GMT minus 7
% hours) or Email: Joe.Loughry@cs.du.edu or mailto:Joe.Loughry@gmail.com
%

\documentclass[conference]{IEEEtran}

\usepackage{cite}

\usepackage[english,british]{babel}
\usepackage{graphicx}
\usepackage{siunitx}
\usepackage[obeyspaces,hyphens]{url}
\newcommand{\URL}[1]{$\langle$\url{#1}$\rangle$}
\usepackage[plainpages=false,pdfpagelabels]{hyperref}

\hyphenation{op-tical net-works semi-conduc-tor}

\begin{document}

\title{The optical \textsc{tempest} principles in the design of a new product
}

\author{\IEEEauthorblockN{Joe Loughry}
\IEEEauthorblockA{University of Denver \\
Denver, Colorado 80208 USA \\
Email: joe.loughry@cs.du.edu}
\IEEEauthorblockA{and the Netoir company \URL{https://netoir.com}}
}

\maketitle

\begin{abstract}
	Research on optical TEMPEST has exploded since 2002 when the first papers
appeared from widely separated locations within a week of each other. In the
intervening time, vulnerabilities have evolved along with hardware, and
several new threat vectors have appeared. Although the supply chain ecosystem
of Ethernet has reduced the vulnerability of billions of devices through use
of standardised PHY chips, other recent trends including High Frequency
Trading (HFT) in the financial sector, the Internet of Things (IoT), and
inexpensive drones have made it relevant again.

\end{abstract}

\section{Introduction}

Since the publication sixteen years ago of two papers on optical
\textsc{tempest}, open sources on security of compromising emanations, and
side channels in general, has exploded. Before 1996, only a handful of papers
had been published on the subject \cite{Smulders1990}, the most famous being
van Eck on RF emanations from CRT displays \cite{vanEck1985}. But following
Kocher's seminal 1996 paper on side channel attacks \cite{Kocher1996}---not
to forget earlier but less scientific reports \cite{Wright1987} dating back
a hundred years---two papers appeared within the same week on complementary
aspects of optical emanations \cite{Kuhn2002,Loughry2002a}.


Cited $n$ times.
Simultaneous discovery with Markus Kuhn of Cambridge; NSA story (in a
footnote), LCD monitors.

The first part of this paper is a survey of further work done by others since
the publication of \cite{Loughry2002a} in 2002. This is followed by a
description of the design process for a new product according to these
principles.

\section{USB}

Universal Serial Bus (USB) devices

\section{PHY chips}

pulse stretchers (references)

\section{HFT}

High Frequency Trading (HFT)

\section{Design of a new product with optical \textsc{tempest} principles
in mind}

According to these hard-won principles.

\subsection{Simple photoconductive photodiode design}

The photodiode circuit is shown in Figure \ref{figure:photodiode_pullup}.
The photodiode operates in reverse bias (photoconductive) mode for two
reasons: speed, and simplicity of implementation. The same photodiode
operated in photovoltaic mode would be more sensitive to very low level
signals and have a lower dark current, but would require a transimpedance
amplifier for current-to-voltage conversion, which would make the design
more complicated and thereby more difficult to evaluate for security.

% Use [!t] for figures in IEEEtrans papers..
\begin{figure}[!t]
    \centering
	\includegraphics[height=2in]{graphics/photodiode_pullup_and_GPIO_protection.png}
	\caption{The photodiode circuit is purposely made as simple as possible
        for transparency of implementation; the \SI{10}{\kilo\ohm} pull-up
        resistor for reliability and a \SI{100}{\ohm} series resistor to
        protect the general-purpose IO (GPIO) pin from being shorted to
        ground in case it were accidentally set to output a HIGH logic level
        at the same time the photodiode was illuminated.}
	\label{figure:photodiode_pullup}
\end{figure}

Risk of acoustic information leakage from acousto-optic modulator or MMD
modulator. Cooling fan speed modulation acoustic information leakage risk.
Light-tight design. RF, acoustic, vibration, temperature, ELF, power line
conducted emissions: see lit.\ survey in recent papers for a comprehensive
list.

\section{Energy Gapping}

Cite the `energy gap' principles from Clive Robinson.

\section{Ethics}

Cite Wilkins on disclosure and double-edged sword.

% See bare_conf.tex for example of a two-column subfigure in IEEEtrans, and
% for the preferred table format.

\section{Conclusion}

IEEEtrans papers always have a Conclusion section.

\section*{Acknowledgment}

This is the acks.

\section{Methodology}

This is a reference \cite{Loughry2013a}.

\section{Results}

Words\ldots

\section{Interpretation}

Words\ldots

\section{Summary and Conclusion}

Words\ldots

\section{Acknowledgements}

More words\ldots

% trigger a \newpage just before the given reference
% number - used to balance the columns on the last page
% adjust value as needed - may need to be readjusted if
% the document is modified later
%\IEEEtriggeratref{8}
% The "triggered" command can be changed if desired:
%\IEEEtriggercmd{\enlargethispage{-5in}}

\bibliographystyle{IEEEtran}
% argument is your BibTeX string definitions and bibliography database(s)
\bibliography{IEEEabrv,consolidated_bibtex_file}

\end{document}

